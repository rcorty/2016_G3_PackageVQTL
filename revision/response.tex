\documentclass[11pt]{article} 
\usepackage[letterpaper, headheight=100pt]{geometry}
\geometry{verbose,tmargin=3cm,bmargin=2cm,lmargin=2.5cm,rmargin=2.5cm}
\usepackage{amsmath}
\usepackage{amssymb}
\usepackage{amsfonts}
\usepackage{graphicx}
\usepackage{hyperref}
\usepackage{fancyhdr}
\pagestyle{fancy}
\usepackage{xspace}
\newcommand{\eg}{\emph{e.g.}\xspace}
\newcommand{\ie}{\emph{i.e.}\xspace}
%\newcommand{\WV}[2]{\textcolor{red}{#1}\footnote{\textcolor{red}{#2}}}
\newcommand{\WV}[2]{\textcolor{red}{#1\footnote{\textcolor{red}{#2}}}}
\newcommand{\RC}[2]{\textcolor{purple}{#1\footnote{\textcolor{purple}{#2}}}}
\newcommand{\reply}{\ensuremath{\hookrightarrow}\xspace}

\newcommand{\CortyRPaper}{\textcolor{purple}{Corty and Valdar 2018+ [r/vqtl]}\xspace}
\newcommand{\CortyReanalysisPaper}{\textcolor{purple}{Corty \emph{et al} 2018+}\xspace}
\newcommand{\CortyMethodsPaper}{\textcolor{purple}{Corty and Valdar 2018+ [BVH]}\xspace}


% Bibliography
\usepackage{natbib} \bibpunct{(}{)}{;}{author-year}{}{,}
\bibliographystyle{genetics}
%\addto{\captionsenglish}{\renewcommand{\refname}{Literature Cited}}
\setlength{\bibsep}{0.0pt}

\usepackage{color}
\usepackage[usenames,dvipsnames]{xcolor}

\newcommand{\alert}[1]{\textcolor{red}{#1}}
\usepackage[left,running,mathlines]{lineno}
\renewcommand\linenumberfont{\color{gray}\normalfont\tiny\sffamily}
\usepackage{listings}

\newcommand{\auassign}[2]{\mbox{}\marginpar{\footnotesize\textbf{#1}\newline \textcolor{red}{#2}}}

\lhead{Manuscript ID: G3/2018/200159 \\ Title: ``\texttt{vqtl}: An \texttt{R} package for Mean-Variance QTL Mapping''}
\chead{\bfseries Response to Reviewer Comments \\ \mbox{}\\ }
\rhead{\today\\William Valdar}
\title{BVH: Response To Editors/Reviewers}

\newcounter{ReviewerNum}
\setcounter{ReviewerNum}{1}

\newcounter{QuestionNum}[ReviewerNum]
\setcounter{QuestionNum}{0}

\newcounter{EditorNum}
\setcounter{EditorNum}{1}

\newcounter{EdQuestionNum}[EditorNum]
\setcounter{EdQuestionNum}{0}

\newcommand{\ReviewerQuestion}[1]{
  \vspace{5pt}\goodbreak
  \noindent\fbox{Reviewer:} #1
  \normalfont\par
}
\newcommand{\EditorQuestion}[1]{
  \vspace{5pt}\goodbreak
  \noindent\fbox{Editor:} #1
  \normalfont\par
}
\newcommand{\Response}[1]{
  \goodbreak
  \textcolor{blue}{#1}
  \normalfont\par
}

\begin{document}
\reversemarginpar

\section*{Editor Comments}

\subsection*{Multi-manuscript Issues (identical across responses to reviewers)}

\EditorQuestion{
  Now that you have all three manuscripts potentially going forward for publication in G3, they can be more tightly linked.
}
\Response{
  We have updated the manuscripts such that each one refers to the other two in the introduction and discussion.
}

\EditorQuestion{
   For instance, one of the other papers could provide the benchmark data that is required for [the software] manuscript.
   So the software paper can include those files and refer to the other manuscript for details.
}
\Response{
  We have incorporated this recommendation, using the dataset from Kumar et al. 2013 that was used in one of the case studies in the manuscript titled ``Mean-Variance QTL Mapping Identifies Novel QTL for Circadian Activity and Exploratory Behavior in Mice''.
  The time required to analyze this dataset, according to number of computer cores available and number of permutations desired, is described in the ``Performance Benchmarks'' section of the software paper and the relevant files are included in its Zenodo repository.
}



\subsection*{Issues Specific to this Manuscript}

\EditorQuestion{
  As suggested by Reviewer 1 it will be very useful to have a benchmark study with times/memory requirements as a function of number of markers and sample size.
  This study will give those interested on using the package an understanding of how this package would scale.
  It will be important to include in the discussion also elements related to scalability: is the package suitable for very large-p/large-n? 
}
\Response{
  We have added this information in the ``Performance Benchmarks'' section. 
  See detailed response to Reviewer 1 comment 3. 
  Note that this has lengthened the paper somewhat, but we can move plots to supplemental if that would be preferred.
}


\EditorQuestion{
  There is a very rich literature in the study of genetic control in environmental variance in both plant and animal breeding as well as model organisms.
  This literature should be mentioned. This comment may also apply to some of the companion papers.
  [Citation recommendations provided by the editor.]
}
\Response{
  Thank you for these suggestions. 
  This literature was foundational toward the development of the work presented in this manuscript, and we agree with the editor that it should be cited in the introduction.
  We have added a paragraph to the introduction that describes how recognizing the heritability of environmental variance is a necessary step toward vQTL and mvQTL mapping (``It has long been recognized\dots'').
}

% Mulder HA, Bijma P, Hill WG. Selection for Uniformity in Livestock by Exploiting Genetic Heterogeneity of Residual Variance. Genet Sel Evol. 2008;40:37-59. 

% Hill WG, Zhang XS. Effects on Phenotypic Variability of Directional Selection Arising Through Genetic Differences in Residual Variability. Genet Res (Camb). 2004;83:121-132. 

% Yang Y, Schön CC, Sorensen D. The genetics of environmental variation for dry matter grain yield in maize. Genet Res (Camb). 2012;94:113-119. 

% Ibáñez N, Moreno A, Nieto B, Piqueras P, Salgado C, Gutierrez JP. Genetic Parameters Related to Environmental Variability of Weight of Mice; Signs of Correlated Canalised Response. Genet Sel Evol. 2008;40:279-293. 

% Ibáñez N, Varona L, Sorensen D, Noguera JL. A study of heterogeneity of environmental variance for slaughter weight in pigs. Animal. 2007;2:19-26. 

% Sørensen P, de los Campos G, Morgante F, Mackay TFC, Sorensen D. Genetic Control of Environmental Variation of Two Quantitative Traits of Drosophila melanogaster Revealed by Whole-Genome Sequencing. Genetics. 2015;201(2):487-497. doi:10.1534/genetics.115.180273. 

\section*{Reviewer 1 Comments}

\ReviewerQuestion{
  The authors mention ``in almost all traits of interest in human health and disease''.
  The detection of QTL has also been widely used in plant and animal populations.
}
\Response{
  The reviewer is quite correct. Rather than add to this sentence, we have removed the explicit reference to health and disease, recognizing that the introduction started too broadly. Instead we have included references to heritability of variance that rely heavily on studies of model organisms, livestock and crops.
}

\ReviewerQuestion{
  The authors used a set of simulated data with 3 chromosomes and 11 equally spaced markers.
  I would like to see an example with a larger genotype data set.
}
\Response{
    Our use of an unrealistically-small genome is to most clearly illustrate the standard workflow.
%    But we are encouraged by the reviewer's enthusiasm to see the package applied to real datasets.
    Since this manuscript has been submitted as part of a three-paper companionship where each of the other papers contains analyses of three full sets of real data using the R package, we feel that using a reduced dataset to demonstrate the package functions is reasonable in this instance.
    Nonetheless, we note that a full, real dataset is used in the newly added performance benchmark, as described below.
}


\ReviewerQuestion{
  The CPU time is indicated for the data set of the example.
  I would like to have the CPU time information in a more realistic data set.
  Perhaps authors can use a free data set from the web.
}
\Response{
  Thank you for this recommendation.
  The authors agree that including this information will make the paper more useful to geneticists who are considering using the package --- we have added two sets of benchmarks.
}
\Response{
  First, we demonstrate the performance of the package according to the number of computer cores used and the number of permutations conducted.
  For this set of benchmarks, we used the intercross from Kumar et. al 2013, which we reanalyzed in \CortyReanalysisPaper.
  This intercross population has 244 mice and was genotyped at 93 markers.
  After applying the hidden Markov model in \texttt{R/qtl} with \texttt{step.size} of 2, the number of loci to be tested was 582.
  As recommended by the editor, this example helps tie together the companions, because the results of this analysis are presented in the manuscript titled ``Mean-Variance QTL Mapping Identifies Novel QTL for Circadian Activity and Exploratory Behavior in Mice''.
  This addition can be found on page 4.
}
\Response{
  Second, we demonstrate the performance of the package according to the size of the mapping population and the number of genetic markers, holding constant the number of permutations at 1000 and the number of cores at 32.
  For this set of benchmarks, we used crosses simulated with \texttt{R/qtl} so we could control the population size and number of genetic markers.
  This addition can be found on page 4.
}

\ReviewerQuestion{
  In the simulation, the authors should indicate the magnitude of the genetic variance explained by the QTL.
}
\Response{
  Thank you for this recommendation.
  We have added this information to the manuscript and it now appears on page 2.
}


\ReviewerQuestion{
  I think the authors should also indicate if the software is capable in terms of computing power to analyze the data from SNP genotyping devices or even sequencing.
}

\Response{
  Thank you for this recommendation.
  In short, the workflow presented here, which relies on permutation to assess statistical significance, would not be practical with a dense genotyping array with $\textgreater$ 10,000 markers, such as the MegaMUGA ($\approx$ 78,000 markers) GigaMuga ($\approx$ 150,000 markers), or the Mouse Diversity Genotyping array ($\approx$ 623,000 markers).
  We have addressed this question in the ``Performance Benchmarks'' section, starting on page 4 and continuing onto page 5.
}


\end{document}

